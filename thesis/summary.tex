\section{论文工作总结}
Android平台的应用, 无论是正常的还是恶意的, 都在利用不断变化的技术来对抗应用分析, 然而对应用进行分析又是辨别出恶意应用比不可少的环节, 因此应用开发者想尽办法隐藏应用的真实行为, 应用分析者想尽办法找出应用隐藏的行为就成了一场持续的对抗\juhao 学习Android系统的设计原理和现有技术则是加入这场对抗十分重要的部分, 在充分理解已有知识的基础上, 才能产生新的知识\juhao 本文将主要工作重心放在学习Android系统和已有的动态分析技术上, 并在新的Android系统上进行了部分实践, 具体来说, 论文的主要工作内容如下:

	 研究了Android系统的整体架构, Android应用开发相关知识,  研究了国内外关于Android平台动态分析的多项技术成果并总结了对抗应用分析的常用技术和实现动态分析常用的技术和工具, 分析了其优势和劣势\juhao 
	 
	 深入源代码分析了Android8.1系统中ART虚拟机的设计和运行机制, 在3个重要方面, 即应用的启动\dunhao 类的加载和方法的执行方面给出了关键调用图\juhao 
	 
	 利用上述研究的成果设计和实现了一个Android应用的动态行为捕获系统, 并实现了简单的脱壳功能\juhao 对本文实现的系统进行了详细的测试, 验证了设计方案的可行性和有效性\juhao 
	 
\section{进一步工作展望}
本文对于Android系统, Andriod平台动态分析技术及其相关技术(如调试技术)的认识还不够深刻, 并且
本文所有构建的系统只是对设计思路的简单测试, 并不是一个成熟的高可用性的系统, 因此未来的
未来的工作分为两部分, 第一部分是继续深入研究和测试已有的调试技术和动态分析技术, 深入研究Android系统各个部分的设计和实现原理, 第二部分是完善本系统的设计和实现, 具体来说有以下方面:

本系统对本地函数的监控方案不够好, 存在较严重的稳定性问题, 需要更稳定有效的实现方案\juhao

本系统的缺乏有效的运行时控制方式, 没有提供在系统运行时用户可以控制系统的接口, 需要提供接口来保证灵活性\juhao 

本系统产生的Java调用日志数量众多且不易阅读, 需要更有效的过滤方案来去除无效信息和冗余信息\juhao 
	