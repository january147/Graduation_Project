\section{研究背景与意义}
随着移动互联网和物联网的蓬勃发展, 智能移动终端设备迅速普及\juhao 
截止2018年9月,国内智能手机用户数量已达到7.8亿\upcite{smartphone2018statistics}, 占总人口数的55\%以上\juhao 如此众多的用户极大地促进了移动应用的发展, 2018年的数据显示\upcite{appAmount2018}, 谷歌公司的Google Play上已经有超260万应用软件, 苹果公司的App Store上也有超过200万应用软件可供下载使用\juhao 这些应用软件涵盖了娱乐, 社交, 购物, 出行, 金融服务, 身份服务等等领域, 极大地便利了人们的生活\juhao 但与此同时, 各种服务通过应用软件集中于智能手机使得智能手机与个人隐私, 财产安全甚至人身安全的联系变得更加紧密, 从而不可避免地吸引了大量攻击者开发和传播恶意应用来牟取不正当利用\juhao 

目前市场上的智能移动终端设备运行的移动操作系统几乎均为Android和Apple iOS\upcite{mobileOSMarketshare2019}\juhao 其中Android以其免费, 开源的特点吸引了大量智能手机厂商, 占据了超过75\%的市场份额\upcite{mobileOSMarketshare2019}. 最新数据显示, 在2019年Q1国内智能手机销售量中搭载Android的手机销量占比达78.2\%\upcite{mobileOSSalesMarketshare2019}\juhao 然而, Android本身宽松的权限管理以及开放的应用分发方式使其很容易受到攻击, 加上巨大的市场体量, 造成了绝大多数恶意应用把Android作为攻击目标的局面\juhao 虽然近年来Android的权限管理和安全机制不断加强, 同时工信部各大应用分发平台的监管加强, 一定程度上遏制了恶意应用的发展, 但数据显示\upcite{malwareStat}2018年Android新增恶意软件达800.62万个, 感染用户数近1.13亿, 数量仍然庞大\juhao 另外, 恶意软件的类型也持续朝着多样化隐秘化方向发展,新式的恶意软件通过更加难以分析的加壳和混淆技术隐藏自己的恶意行为, 绕过安全软件的查杀\juhao 因此, Android平台的安全问题依然严峻\juhao 

为了阻止恶意应用被下载运行, 保护用户手机的信息安全, 各大应用分发平台需要能够精确有效地判断开发者提交的应用是否为恶意应用, 而捕获应用的行为是分析一个新应用是否为恶意应用的必要前提\juhao 对应用软件的行为获取方法有两个大类: 静态方式和动态方式\juhao 静态方式即在不运行应用软件的情况下对应用软件内部的资源文件、代码、数据等进行分析, 获取应用的特征, 代码逻辑等; 动态方式则是运行应用软件, 在执行过程中对软件的代码执行路径, 数据访问等进行监控和记录\juhao 在目前Android平台的应用加壳和混淆技术成熟的情况下, 单独的静态分析无法获取包含应用的真正逻辑的代码和数据, 因而无法获取到应用行为, 必须通过动态的方式才能捕获到包含应用真实目的的行为, 获取相应的数据, 从而判断应用是否为恶意应用\juhao 另外, 动态分析还能够在运行中捕获到执行应用真正逻辑的代码和数据(脱壳), 从而结合静态分析揭示更加完整的应用行为\juhao 因此, 对Android平台移动应用的动态行为捕获技术进行研究, 有助于识别和分析隐蔽性越来越强的恶意应用, 从而遏制恶意应用的传播, 提升Android平台的安全性\juhao

\section{国内外研究现状和发展方向}
Android系统从发布至今已有10年, 从2011年开始就有与Android应用动态分析相关的研究成果陆续发表\juhao 同传统PC平台的软件动态分析相比,  

\section{论文主要工作}

\section{论文组织结构}

 