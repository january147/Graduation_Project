\section{研究背景与意义}
随着移动互联网和物联网的蓬勃发展, 智能移动终端设备迅速普及\juhao 
截止2018年9月,国内智能手机用户数量已达到7.8亿\upcite{smartphone2018statistics}, 占总人口数的55\%以上\juhao 如此众多的用户极大地促进了移动应用的发展, 2018年的数据显示\upcite{appAmount2018}, 谷歌公司的Google Play上已经有超260万应用软件, 苹果公司的App Store上也有超过200万应用软件可供下载使用\juhao 这些应用软件涵盖了娱乐, 社交, 购物, 出行, 金融服务, 身份服务等等领域, 极大地便利了人们的生活\juhao 但与此同时, 各种服务通过应用软件集中于智能手机使得智能手机与个人隐私, 财产安全甚至人身安全的联系变得更加紧密, 从而不可避免地吸引了大量攻击者开发和传播恶意应用来牟取不正当利用\juhao 

目前市场上的智能移动终端设备运行的移动操作系统几乎均为Android和Apple iOS\upcite{mobileOSMarketshare2019}\juhao 其中Android以其免费, 开源的特点吸引了大量智能手机厂商, 占据了超过75\%的市场份额\upcite{mobileOSMarketshare2019}. 最新数据显示, 在2019年Q1国内智能手机销售量中搭载Android的手机销量占比达78.2\%\upcite{mobileOSSalesMarketshare2019}\juhao 然而, Android本身宽松的权限管理以及开放的应用分发方式使其很容易受到攻击, 加上巨大的市场体量, 造成了绝大多数恶意应用把Android作为攻击目标的局面\juhao 虽然近年来Android的权限管理和安全机制不断加强, 同时工信部各大应用分发平台的监管加强, 一定程度上遏制了恶意应用的发展, 但数据显示\upcite{malwareStat}2018年Android新增恶意软件达800.62万个, 感染用户数近1.13亿, 数量仍然庞大\juhao 另外, 恶意软件的类型也持续朝着多样化隐秘化方向发展,新式的恶意软件通过更加难以分析的加壳和混淆技术隐藏自己的恶意行为, 绕过安全软件的查杀\juhao 因此, Android平台的安全问题依然严峻\juhao 

为了阻止恶意应用被下载运行, 保护用户手机的信息安全, 各大应用分发平台需要能够精确有效地判断开发者提交的应用是否为恶意应用, 而捕获应用的行为是分析一个新应用是否为恶意应用的必要前提\juhao 对应用软件的行为获取方法有两个大类: 静态方式和动态方式\juhao 静态方式即在不运行应用软件的情况下对应用软件内部的资源文件、代码、数据等进行分析, 获取应用的特征, 代码逻辑等; 动态方式则是运行应用软件, 在执行过程中对软件的代码执行路径, 数据访问等进行监控和记录\juhao 在目前Android平台的应用加壳和混淆技术成熟的情况下, 单独的静态分析无法获取包含应用的真正逻辑的代码和数据, 因而无法获取到应用行为, 必须通过动态的方式才能捕获到包含应用真实目的的行为, 获取相应的数据, 从而判断应用是否为恶意应用\juhao 另外, 动态分析还能够在运行中捕获到执行应用真正逻辑的代码和数据(脱壳), 从而结合静态分析揭示更加完整的应用行为\juhao 因此, 对Android平台移动应用的动态行为捕获技术进行研究, 有助于识别和分析隐蔽性越来越强的恶意应用, 从而遏制恶意应用的传播, 提升Android平台的安全性\juhao

\section{国内外研究现状和发展方向}
Android系统从发布至今已有10年, 目前国内外已有许多Android应用动态分析相关的研究成果发表\juhao 这些成果借鉴了传统PC平台的动态分析方法, 并结合了Android平台的自身特点, 在本地指令层面, 系统调用层面, 本地函数层面, Java指令层面, Java方法层面中部分或全部层面对应用的运行进行跟踪记录, 并在此基础上结合污点传播技术实现了隐私数据泄露的检测功能, 结合对应用加壳混淆机制的研究实现了脱壳和去混淆功能, 给恶意应用的分析提供了许多强有力的工具\juhao 下文介绍了一些有代表性的成果\juhao

Enck William等构建了名为Taintdroid\upcite{taintdroid}的隐私数据跟踪系统\juhao 该系统采用了污点传播技术, 通过修改Android系统Java层与隐私数据获取相关的API给隐私数据添加标记, 通过修改Android Runtime的Dalvik虚拟机运行机制实现了带标记隐私数据在虚拟机内部的透明传播, 通过修改Android系统进程间通信的接口实现了带标记隐私数据跨进程传播, 通过修改Java层文件和网络的API实现记录带标记的隐私数据去向, 从而能够检测到应用泄露隐私数据的行为\juhao 不过该系统有以下局限性: 
1. 没有对应用的所有敏感行为进行监控, 例如拨打电话, 发送短信等\juhao
2. 没有对Native层的函数进行监控, 无法检测到应用通过JNI接口调用自身Native模块泄露隐私数据的行为\juhao
3. 针对特定Android版本, 并且不再支持Android4.3以后的版本使用\juhao
 
Desnos Anthony等构建了名为Droidbox的\upcite{droidbox}动态分析系统\juhao 该系统使用了Taintdroid\upcite{taintdroid}来监控隐私数据泄露, 另外通过修改Android系统源代码中敏感API的方式实现对Java层的电话, 短信, 网络, 文件, Java类动态加载, 加密等API调用的监控, 能够记录应用在Java层的敏感行为\juhao 但该系统只涉及了Java层预定义的敏感API的监控, 没有实现对应用本地函数层调用的监控, 因此无法完整的揭示应用的行为, 另外该系统也针对特定Android版本开发, 并且不支持Android4.1之后的系统使用\juhao

Google公司构建了名为Bouncer\upcite{googlebouncer}的恶意应用检测系统\juhao 该系统利用静态分析的方式识别已知的恶意软件, 同时通过动态分析方式使用虚拟机运行应用并记录其行为, 从而判断其是否为恶意应用\juhao

Yan Lok Kwong等构建了名为DroidScope\upcite{droidscope}的动态分析系统\juhao 该系统通过修改运行Android系统的qemu虚拟机以及运行的Android系统来实现, 提供了对本地指令层面和Java指令层面的


\section{论文主要工作}
本文分析了目前已有的一些安卓系统应用行为监测系统的实现方式和优缺点, 并且通过hook技术以及对安卓8.1源代码的修改设计和实现了一个运行于Nexus 5x(Google的一款智能手机)的高性能应用动态行为捕获系统. 该系统能够捕获到Java层的所有方法调用以及Native层的重要函数调用, 并且支持动态地调整需要监控的目标方法(Java层)和函数(Native层). 本文使用常用应用对该系统进行了测试, 结果显示与同样能捕获到所有java层方法调用的Android Device Monitor相比本系统的性能开销明显更低.

\section{论文组织结构}
根据本文研究的特点, 本文的内容按如下方式组织:

第一章为绪论, 主要说明了本文课题的研究背景, 研究意义, 简述了国内外对本文课题的研究现状及发展方向, 介绍了本文的主要工作内容和文章组织结构\juhao 

第二章为背景技术介绍, 主要讲述了Android系统的基本架构, Android应用的基本结构, Android平台的常用动态分析技术和应用保护技术, Android Runtime的运行机制和Frida框架\juhao

第三章为系统设计和实现, 详细说明了本文提出的监控系统的原理与组成结构\juhao

第四章为实验与结果分析, 描述了实验环境, 实验方法并对实验结果进行了分析和总结\juhao

第五章为总结与展望, 主要是整理本文所做的工作,并简要分析了本文提出系统的局限性和改进方案\juhao

 